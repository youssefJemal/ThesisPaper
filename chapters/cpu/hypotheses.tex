\section{Hypotheses}
The following hypotheses guide our investigation of CPU performance degradation:

\begin{enumerate}[label=H\arabic*:, leftmargin=*, align=left]
    \item Overhead introduced by virtualization, such as hypervisor overhead, is expected 
    to be minimal and practically insignificant. The substantial improvements introduced 
    by hardware-assisted virtualization and the Nitro System support this assumption.

    \item For hosts that support \ac{SMT}, physical core co-location between the 
    co-tenants will likely create measurable performance degradation. If the vCPUs of the 
    test node are mapped to several physical cores, we should witness a gradual performance 
    loss, as busy neighbors start sharing these cores with the test node.

    \item Frequency scaling can be a reason for performance degradation.  
    When the test node runs alone on the dedicated host, it should operate at peak frequency and 
    achieve optimal performance. As busy neighbors are added, the processor's average frequency
    is likely to decrease due to power and temperature limits, leading to performance loss.
\end{enumerate}

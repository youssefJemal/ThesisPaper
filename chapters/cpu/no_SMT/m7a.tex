\subsection{m7a Family}
The m7a host, powered by the AMD EPYC 9R14 processor, utilizes frequency scaling and has SMT disabled, 
making it an ideal candidate for investigating the effect of frequency scaling on degradation. 
We ran the CPU experiment using 4xlarge instances to determine whether any degradation occurs on the 
test node. As table \ref{tab::m7a} shows, we observe a performance loss of 9.15\% by adding busy neighbors.  
The metal instance experiences a similar degradation of about 9.3\%. For the metal instance, the initial 
runtime corresponds to running \textit{cpu\_burn} with 16 threads and final runtime with 192 threads. \\
Since virtual machines do not have read access to the frequencies of each individual cores, 
we measured the average CPU frequency of the busy cores on the metal instances.
Busy cores can be identified by their frequencies exceeding 2.6 GHz, which corresponds to
the base clock of the AMD EPYC 9R14 processor. With 16 busy threads, we measured an average 
frequency of 3.7 GHz, which represents the peak frequency. 
However, at 192 busy thread, we observe a drop to 3.4 GHz in the average frequency (8.1\%). \\ 
These findings strongly suggest that the observed performance degradation is caused by the 
frequency scaling mechanism: As more physical cores become busy, the average frequency decreases 
to adhere with power consumption and temperature limits. This explanation also extends to the 
previous experiments, since all hosts supported 
frequency scaling. 
\begin{table}[h]
\centering
\small
\begin{tabular}{|l|c|c|c|}
\hline
 & \textbf{test node (4xlarge)} & \textbf{m7a.metal} & \textbf{Average frequency} \\
\hline
\textbf{Initial runtime (s)} & 6.642 & 6.66 & 3.7 GHz \\
\hline
\textbf{Final runtime (s) } & 7.25 & 7.28 & 3.4 GHz \\
\hline
\textbf{Degradation} & 9.15\% & 9.3\% & 8.1\% \\
\hline
\end{tabular}
\caption{Runtime and frequency comparison for the test node (4xlarge) and the m7a.metal instance}
\label{tab::m7a}
\end{table}


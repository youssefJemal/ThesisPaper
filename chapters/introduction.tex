\chapter{Introduction}\label{chapter:introduction}

\ac{IaaS} is a cloud computing model that provides customers with access to computing 
resources such as servers, networking, and virtualization. Cloud vendors in general and \ac{AWS} in 
particular abstract the physical placement of virtual machines, providing users with limited 
transparency about how many tenants are sharing the underlying hardware. This information can be crucial as 
this co-residency can result in significant performance degradation across different resources such as 
CPU, memory, network, and storage I/O \cite{characterizing_public} \cite{mitigating_resource}. \\
In 2017, \ac{AWS} launched the Nitro system \cite{nitro_whitepaper}, which enhances virtualization 
in their data centers by offloading most virtualization tasks to dedicated hardware.
However, contention can still occur due to other factors such as \ac{SMT}, Network Interface
Card queuing, and other system-level bottlenecks. \\
This thesis provides empirical evidence of the 
maximum possible performance degradation on AWS and investigates its primary causes. Understanding 
these performance impacts is important for customers when selecting a suitable instance type 
for their workloads, as co-located performance degradation may cause them to receive less value 
than paid for. 
Throughout this thesis, we leverage different benchmarking tools to assess the extent of resource 
contention. We ran identical resource-intensive benchmarks 
in parallel across VMs residing on the same dedicated host. We primarily utilized 5th and 6th 
general-purpose \ac{AWS} EC2 instances that leverage the \ac{AWS} Nitro system. We assess
CPU performance degradation across various vendors, namely Intel, Graviton, and AMD, and analyze 
its contributing factors. Furthermore, we investigate network resource 
contention, which is important since \ac{AWS} does not provide exact specifications like other 
resources, such as CPU and memory. We quantify the degradation that can occur both on throughput and 
latency across co-located EC2 instances.\\ 
The thesis is structured as follows: We start by discussing related work in Section 2. 
Section 3 introduces the key concepts required to understand this work.
Section 4 provides information about our testing infrastructure. 
Section 5 analyzes CPU contention across 5th and 6th EC2 generations and identifies its causes. In Section 6, 
we examine network I/O contention and explore its manifestation across throughput and latency, 
before we conclude in section 7. 


\begin{comment}
Contention can still occur due to other factors such as \ac{SMT}, Network Interface
Card queuing, and other system-level bottlenecks. \\

Understanding the extent of VM co-location is critical for cloud users, as it directly
affects the performance and efficiency of deployed applications.

\end{comment}
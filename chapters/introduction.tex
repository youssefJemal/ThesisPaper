\chapter{Introduction (\checkmark)}\label{chapter:introduction}

\ac{IaaS} is a cloud computing model that provides customers with access to computing 
resources such as servers, networking and virtualization. Cloud vendors in general and \ac{AWS} in particular 
abstract the physical placement of the virtual machine providing users with limited transparency to how 
many tenants are sharing the underlying hardware. This information can be crucial as 
this co-residency can result in significant performance degradation across different resources such as 
CPU, memory, network and storage I/O \cite{characterizing_public} \cite{mitigating_resource}. 
Although virtualization technology has undergone major improvements in 
the past decades, contention can still occur due to other factors such as \ac{SMT}, Network Interface
Card queuing, and other system-level bottlenecks. To address the issue of unwanted resource 
contention, Amazon Web Services introduced dedicated hosts \cite{dedHost}, which are physical servers that are fully 
dedicated to the customer,  allowing users to deploy virtual machines with complete transparency over their
placement. Another potential solution are the spread placement groups, which ensure that the VMs are placed 
on different physical servers. It can be
particularly useful for highly parallel workloads, where the nodes have a similar workload nature
(network- or CPU-intensive). However, this approach only mitigates internal resource contention and is 
limited to a small number of EC2 instances in the same availability zone \cite{spread}. \\
In this paper, we leverage different benchmarking tools to assess the extent of the performance degradation 
that can occur because of VM co-location. We are particularly interested in quantifying the worst possible 
degradation that can happen by running identical resource-intensive benchmarks in parallel across VMs 
residing on the same dedicated host. We primarily utilized 5th and 6th general-purpose \ac{AWS} EC2 VM 
instances that are built around the \ac{AWS} Nitro system and run on top of the Nitro hypervisor. 
We analyze CPU contention across various CPU vendors namely Intel, Gravtion, and AMD. 
We examine whether and to what extent \ac{SMT} is involved in this degradation. 
Furthermore, we investigate network resource contention, which is interesting since \ac{AWS} does not provide
exact specifications like other resources such as CPU and memory. 
We analyze the degradation that can occur both on throughput and latency across co-located EC2 Instances.\\ 
The thesis is structured as follows: We start by discussing related work in section 2. 
Section 3 introduces the key concepts required to understand this work, namely virtualization and 
Simulatenous Multi Threading. In section 4, we explain the methodology of our experiments
and introduce the different benchmarking tools that were adopted throughout the thesis. 
Section 5 analyzes CPU contention across 5th and 6th EC2 generations and contexualizes the results in 
relation to the CPU architecture. In Section 6, we examine network I/O contention and 
explore its manifestation across throughput and latency. Section 7 concludes our work and summarizes 
its most important findings. 
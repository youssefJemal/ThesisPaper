\chapter{Conclusion}\label{chapter:conclusion}

This thesis contributes empirical evidence of the worst-case impact of VM co-location on AWS instances, 
with a focus on CPU and network I/O.
For CPU, we found that \ac{SMT}-enabled hosts are particularly susceptible to 
performance degradation. Among the evaluated families, m5 and m6a were the most affected, 
with performance loss reaching 13\%.
In contrast, the m6i family exhibited only minor degradation of less than 2\%.
The m5 family also showed unexpected performance degradation even with idle neighbors. \
Furthermore, our results suggest that AWS enforces vCPU isolation between virtual machines, 
that has a negative impact on the baseline performance of these instances. \\ 
These findings highlight that even with continued improvement to virtualization, considerable 
resource degradation is still possible. Hardware features such as \ac{SMT}
can introduce non-trivial performance penalties. It's crucial that AWS
customers consider these characteristics when choosing an instance type for a CPU-intensive workload. 
m5 instance users seem to be at a disadvantage, since CPU performance is degraded
solely by the presence of neighbors. However, Graviton instances, such as the m6g family,
provide a (nearly) complete isolation between tenants, offering consistent and reliable CPU performance.\\ 
The thesis also investigated network I/O contention on the m5 family and its effect 
on throughput and latency. Throughput degradation can reach 30\% for large instances, relative 
to their baseline bandwidth. For latency, the measurements had high variability but we discovered that 
a degradation of 217\% is possible. \\
Throughput and latency degradation are mostly dependent on the bandwidth of the underlying 
dedicated host. For network-intensive applications, users should opt for network-optimized
instances that provide higher bandwidth. Otherwise, co-location effects could severely impact 
application performance. 
\PassOptionsToPackage{table,svgnames,dvipsnames}{xcolor}

\usepackage[a-2u]{pdfx} % generate PDF/A: archival compliant, self-contained pdf
\usepackage[utf8]{inputenc}
\usepackage[T1]{fontenc}
\usepackage[sc]{mathpazo}
\usepackage{diagbox}
\usepackage{float}
\usepackage{nameref}
\usepackage[ngerman,american]{babel}
\usepackage[autostyle]{csquotes}
\usepackage{comment}
\usepackage[%
  backend=biber,
  url=false,
  style=numeric,
  maxnames=4,
  minnames=3,
  maxbibnames=99,
  giveninits,
  uniquename=init]{biblatex} % TODO: adapt citation style
\usepackage{graphicx}
\usepackage{scrhack} % necessary for listings package
\usepackage{listings}
\lstset{
  language=C,
  basicstyle=\ttfamily\small,
  keywordstyle=\color{blue},
  commentstyle=\color{gray},
  stringstyle=\color{orange},
  numbers=left,
  numberstyle=\tiny,
  stepnumber=1,
  numbersep=5pt,
  showstringspaces=false,
  breaklines=true,
  frame=single,
  captionpos=b,
  tabsize=4
}
\usepackage{lstautogobble}
\usepackage{tikz}
\usepackage{pgfplots}
\usepackage{amssymb}
\usepackage{pgfplotstable}
\usepackage{booktabs}
\usepackage[final]{microtype}
\usepackage[font=small,justification=centering]{caption}
\usepackage[printonlyused]{acronym}
\usepackage{ifthen}
\usepgfplotslibrary{groupplots}
\pgfplotsset{
        compat=1.3,
        % create a custom style for the boxplots
        my boxplot style/.style={
            boxplot,
            draw=black,
            solid,
            fill=white,
            % -------------------------------------------------------------
            % add your desired mark symbol and mark style here
            mark=*,
            every mark/.append style={
                fill=gray,
            },
            % -------------------------------------------------------------
        },
    }
\pgfplotsset{compat=1.17}
\usepackage{pgfplotstable}
\usetikzlibrary{pgfplots.statistics}

\hypersetup{hidelinks} % removes colored boxes around references and links

% for fachschaft_print.pdf
\makeatletter
\if@twoside
	\typeout{TUM-Dev LaTeX-Thesis-Template: twoside}
\else
	\typeout{TUM-Dev LaTeX-Thesis-Template: oneside}
\fi
\makeatother

\addto\extrasamerican{
	\def\lstnumberautorefname{Line}
	\def\chapterautorefname{Chapter}
	\def\sectionautorefname{Section}
	\def\subsectionautorefname{Subsection}
	\def\subsubsectionautorefname{Subsubsection}
}

\addto\extrasngerman{
	\def\lstnumberautorefname{Zeile}
}

% Themes
\ifthenelse{\equal{\detokenize{dark}}{\jobname}}{%
  % Dark theme
  \newcommand{\bg}{black} % background
  \newcommand{\fg}{white} % foreground
  \usepackage[pagecolor=\bg]{pagecolor}
  \color{\fg}
}{%
  % Light theme
  \newcommand{\bg}{white} % background
  \newcommand{\fg}{black} % foreground
}

\bibliography{bibliography}

\setkomafont{disposition}{\normalfont\bfseries} % use serif font for headings
\linespread{1.05} % adjust line spread for mathpazo font

% Add table of contents to PDF bookmarks
\BeforeTOCHead[toc]{{\cleardoublepage\pdfbookmark[0]{\contentsname}{toc}}}

% Define TUM corporate design colors
% Taken from http://portal.mytum.de/corporatedesign/index_print/vorlagen/index_farben
\definecolor{TUMBlue}{HTML}{0065BD}
\definecolor{TUMSecondaryBlue}{HTML}{005293}
\definecolor{TUMSecondaryBlue2}{HTML}{003359}
\definecolor{TUMBlack}{HTML}{000000}
\definecolor{TUMWhite}{HTML}{FFFFFF}
\definecolor{TUMDarkGray}{HTML}{333333}
\definecolor{TUMGray}{HTML}{808080}
\definecolor{TUMLightGray}{HTML}{CCCCC6}
\definecolor{TUMAccentGray}{HTML}{DAD7CB}
\definecolor{TUMAccentOrange}{HTML}{E37222}
\definecolor{TUMAccentGreen}{HTML}{A2AD00}
\definecolor{TUMAccentLightBlue}{HTML}{98C6EA}
\definecolor{TUMAccentBlue}{HTML}{64A0C8}

% Settings for pgfplots
\pgfplotsset{compat=newest}
\pgfplotsset{
  % For available color names, see http://www.latextemplates.com/svgnames-colors
  cycle list={TUMBlue\\TUMAccentOrange\\TUMAccentGreen\\TUMSecondaryBlue2\\TUMDarkGray\\},
}

% Settings for lstlistings
\lstset{%
  basicstyle=\ttfamily,
  columns=fullflexible,
  autogobble,
  keywordstyle=\bfseries\color{TUMBlue},
  stringstyle=\color{TUMAccentGreen},
  captionpos=b
}
